\documentclass{beamer}
\mode<presentation>
\usepackage{amsmath}
\usepackage{amssymb}
%\usepackage{advdate}
\usepackage{adjustbox}
\usepackage{subcaption}
\usepackage{enumitem}
\usepackage{multicol}
\usepackage{listings}
\usepackage{xcolor}
 
\definecolor{codegreen}{rgb}{0,0.6,0}
\definecolor{codeblue}{rgb}{0.8,0.3,0.3}
\definecolor{codegray}{rgb}{0.5,0.5,0.5}
\definecolor{codepurple}{rgb}{0.58,0,0.82}
\definecolor{backcolour}{rgb}{0.99,0.99,0.99}
 
\lstdefinestyle{mystyle}{
    backgroundcolor=\color{backcolour},  
    commentstyle=\color{codegreen},
    keywordstyle=\color{codeblue},
    numberstyle=\tiny\color{codegray},
    stringstyle=\color{codepurple},
    basicstyle=\ttfamily\footnotesize,
    breakatwhitespace=false,        
    breaklines=true,                
    captionpos=b,                   
    keepspaces=true,                
    showspaces=false,               
    showstringspaces=false,
    showtabs=false,                 
    tabsize=2
}
 
\lstset{style=mystyle}
\usepackage{url}

\def\UrlBreaks{\do\/\do-}
\usetheme{Warsaw}
\usecolortheme{structure}


{
  \leavevmode%
  \hbox{%
  \begin{beamercolorbox}[wd=\paperwidth,ht=2.25ex,dp=1ex,right]{author in head/foot}%
    \insertframenumber{} / \inserttotalframenumber\hspace*{2ex} 
  \end{beamercolorbox}}%
  \vskip0pt%
}
\setbeamertemplate{navigation symbols}{}

\providecommand{\nCr}[2]{\,^{#1}C_{#2}} % nCr
\providecommand{\nPr}[2]{\,^{#1}P_{#2}} % nPr
\providecommand{\mbf}{\mathbf}
\providecommand{\pr}[1]{\ensuremath{\Pr\left(#1\right)}}
\providecommand{\qfunc}[1]{\ensuremath{Q\left(#1\right)}}
\providecommand{\sbrak}[1]{\ensuremath{{}\left[#1\right]}}
\providecommand{\lsbrak}[1]{\ensuremath{{}\left[#1\right.}}
\providecommand{\rsbrak}[1]{\ensuremath{{}\left.#1\right]}}
\providecommand{\brak}[1]{\ensuremath{\left(#1\right)}}
\providecommand{\lbrak}[1]{\ensuremath{\left(#1\right.}}
\providecommand{\rbrak}[1]{\ensuremath{\left.#1\right)}}
\providecommand{\cbrak}[1]{\ensuremath{\left\{#1\right\}}}
\providecommand{\lcbrak}[1]{\ensuremath{\left\{#1\right.}}
\providecommand{\rcbrak}[1]{\ensuremath{\left.#1\right\}}}
\theoremstyle{remark}
\newtheorem{rem}{Remark}
\newcommand{\sgn}{\mathop{\mathrm{sgn}}}
\providecommand{\abs}[1]{\left\vert#1\right\vert}
\providecommand{\res}[1]{\Res\displaylimits_{#1}} 
\providecommand{\norm}[1]{\lVert#1\rVert}
\providecommand{\mtx}[1]{\mathbf{#1}}
\providecommand{\mean}[1]{E\left[ #1 \right]}
\providecommand{\fourier}{\overset{\mathcal{F}}{ \rightleftharpoons}}
%\providecommand{\hilbert}{\overset{\mathcal{H}}{ \rightleftharpoons}}
\providecommand{\system}{\overset{\mathcal{H}}{ \longleftrightarrow}}
	%\newcommand{\solution}[2]{\textbf{Solution:}{#1}}
%\newcommand{\solution}{\noindent \textbf{Solution: }}
\providecommand{\dec}[2]{\ensuremath{\overset{#1}{\underset{#2}{\gtrless}}}}
\newcommand{\myvec}[1]{\ensuremath{\begin{pmatrix}#1\end{pmatrix}}}
\let\vec\mathbf

\lstset{
%language=C,
frame=single, 
breaklines=true,
columns=fullflexible
}

\numberwithin{equation}{section}

\title{EE2101-Control Systems\\Ass-1,Problem 6}
\author{A PRASHANTH\\EE19BTECH11003}

\date{\today} 
\begin{document}

\begin{frame}
\titlepage
\end{frame}



\begin{frame}
\tableofcontents
\end{frame}
\section{Problem}

\begin{frame}
\frametitle{Problem }
Use MATLAB and the Symbolic Math Toolbox to find the inverse Laplace transform of the following frequency functions :
\begin{equation}
    a.\quad  G(s) = \frac{(s+5)(s^2+3s+10)}{(s+3)(s+4)(s^2+2s+100)}
\end{equation}
\begin{equation}
    b.\quad G(s) = \frac{(s^3+4s^2+2s+6)}{(s+8)(s^2+8s+3)(s^2+5s+7)}
\end{equation}

\vspace{2.5cm}⁣P.S. Here we are using Python instead of MATLAB


\end{frame}

%\subsection{Literature}
\section{Solution}


\begin{frame}
\subsection{Solution for a.}
\frametitle{a. Solution}
%\framesubtitle{Literature}
Firstly let us take function a. \quad i.e.
\begin{equation}
    G(s) = \frac{(s+5) (s^2+3s+10)}{(s+3)(s+4)(s^2+2s+100)}
\end{equation}
Inverse Laplace transform of G(s):
\begin{equation}
    L^{-1}\left\{\frac{\left(s+5\right)\left(s^2+3s+10\right)}{\left(s+3\right)\left(s+4\right)\left(s^2+2s+100\right)}\right\}
\end{equation}
Now let us take partial fraction of
\begin{equation}
    \frac{(s+5) (s^2+3s+10)}{(s+3)(s+4)(s^2+2s+100)} \quad is
\end{equation}
\begin{equation}
    \implies \frac{s^3+8s^2+25s+50}{\left(s+3\right)\left(s+4\right)\left(s^2+2s+100\right)}
\end{equation}

\end{frame}
\begin{frame}{Solution}
    Now create partial fraction template using the denominator,
\begin{equation}
    \left(s+3\right)\left(s+4\right)\left(s^2+2s+100\right)
\end{equation}
i.e
\begin{equation*}
    \frac{s^3+8s^2+25s+50}{\left(s+3\right)\left(s+4\right)\left(s^2+2s+100\right)}=\frac{a_0}{s+3}+\frac{a_1}{s+4}+\frac{a_3s+a_2}{s^2+2s+100}
\end{equation*}
\mathrm{Multiply\:equation\:by\:the\:denominator,}
\begin{equation*}
   \implies \frac{\left(s^3+8s^2+25s+50\right)\left(s+3\right)\left(s+4\right)\left(s^2+2s+100\right)}{\left(s+3\right)\left(s+4\right)\left(s^2+2s+100\right)}=\\ $\frac{a_0\left(s+3\right)\left(s+4\right)\left(s^2+2s+100\right)}{s+3}+\frac{a_1\left(s+3\right)\left(s+4\right)\left(s^2+2s+100\right)}{s+4}+\frac{\left(a_3s+a_2\right)\left(s+3\right)\left(s+4\right)\left(s^2+2s+100\right)}{s^2+2s+100}$
\end{equation*}

\end{frame}
\begin{frame}{Solution}
By simplifying above equation,

\begin{equation*}
     s^3+8s^2+25s+50=a_0\left(s+4\right)\left(s^2+2s+100\right)
     \\
     +a_1\left(s+3\right)\left(s^2+2s+100\right)+\left(a_3s+a_2\right)\left(s+3\right)\left(s+4\right)
\end{equation*}
\vspace{2mm}

Solve the unknown parameters by plugging the\:real roots of the denominator  -3, -4 \\
For the root denominator -3:\quad a_0= \frac{20}{103} \\
For the root denominator -4:\qquad a_1=-\frac{7}{54} \\
\begin{equation*}
    \therefore a_0=\frac{20}{103},\:a_1=-\frac{7}{54}
\end{equation*}


\end{frame}
\begin{frame}{Solution}
    Plug\:in\:the\:solutions\:to\:the\:known\:parameters,
\begin{equation*}
    s^3+8s^2+25s+50=\frac{20}{103}\left(s+4\right)\left(s^2+2s+100\right)
    \\
    +\left(-\frac{7}{54}\right)\left(s+3\right)\left(s^2+2s+100\right)+\left(a_3s+a_2\right)\left(s+3\right)\left(s+4\right)
\end{equation*}
\vspace{2.5mm}

Expand
\begin{equation*}
    s^3+8s^2+25s+50=a_3s^3+\frac{359s^3}{5562}+a_2s^2+7a_3s^2+\frac{2875s^2}{5562}+7a_2s
    \\
    +12a_3s+\frac{20107s}{2781}+12a_2-\frac{350}{9}+\frac{8000}{103}
\end{equation*}
\vspace{2.5mm}

\end{frame}

\begin{frame}{Solution}

\mathrm{Extract\:Variables\:from\:within\:fractions,}
\begin{equation*}
    s^3+8s^2+25s+50=a_3s^3+\frac{359}{5562}s^3+a_2s^2+7a_3s^2+\frac{2875}{5562}s^2+7a_2s
    \\
    +12a_3s+\frac{20107}{2781}s+12a_2-\frac{350}{9}+\frac{8000}{103}
\end{equation*}
\vspace{2.7}

\mathrm{Group\:elements\:according\:to\:powers\:of\:}s,
\begin{equation*}
    1\cdot \:s^3+8s^2+25s+50=s^3\left(a_3+\frac{359}{5562}\right)+s^2\left(a_2+7a_3+\frac{2875}{5562}\right)
    \\
    +s\left(7a_2+12a_3+\frac{20107}{2781}\right)+\left(12a_2+\frac{8000}{103}-\frac{350}{9}\right)
\end{equation*}
\end{frame}
\begin{frame}{Solution}
    Equate the coefficients of similar terms on both sides to create a list of equations
\begin{equation*}
    \begin{bmatrix}12a_2-\frac{350}{9}+\frac{8000}{103}=50\\ 7a_2+12a_3+\frac{20107}{2781}=25\\ a_2+7a_3+\frac{2875}{5562}=8\\ a_3+\frac{359}{5562}=1\end{bmatrix}
\end{equation*}
By solving above equation,
\begin{equation*}
    \therefore a_2=\frac{2600}{2781},\:a_3=\frac{5203}{5562}
\end{equation*}
\end{frame}
\begin{frame}{Solution}
    Plug the solutions to the partial fraction parameters to obtain the final result \\
\therefore a_0=\frac{20}{103},\:a_1=-\frac{7}{54},a_2=\frac{2600}{2781},\:a_3=\frac{5203}{5562} \\ 
\vspace{2mm}
i.e.
\begin{equation*}
    \frac{\frac{20}{103}}{s+3}+\frac{\left(-\frac{7}{54}\right)}{s+4}+\frac{\frac{5203}{5562}s+\frac{2600}{2781}}{s^2+2s+100}
\end{equation*}
Simplify above eqn.
\begin{equation*}
    \frac{20}{103\left(s+3\right)}-\frac{7}{54\left(s+4\right)}+\frac{5203s+5200}{5562\left(s^2+2s+100\right)}
\end{equation*}
\end{frame}
\begin{frame}{Solution}
    \therefore the\:partial\:fractions\:of\:G(s)\:are \\
\begin{equation*}
    L^{-1}\left\{\frac{20}{103\left(s+3\right)}-\frac{7}{54\left(s+4\right)}+\frac{5203s+5200}{5562\left(s^2+2s+100\right)}\right\}
\end{equation*}
After expanding
\begin{equation*}
    =L^{-1}\biggr\{\frac{20}{103\left(s+3\right)}-\frac{7}{54\left(s+4\right)}+\frac{5203}{5562}\cdot \frac{s+1}{\left(s+1\right)^2+99}
\end{equation*}
\begin{equation*}
    -\frac{1}{1854}\cdot\frac{1}{\left(s+1\right)^2+99}\biggr\}
\end{equation*}



\end{frame}
\begin{frame}{Solution}
\mathrm{By\:using\:linearity\:property\:of\:Inverse\:Laplace\:Transform:} \\
\mathrm{For\:functions\:}f\left(s\right),\:g\left(s\right)\mathrm{\:and\:constants\:}a,\:b:\quad L^{-1}\left\{a\cdot f\left(s\right)+b\cdot g\left(s\right)\right\}=a\cdot L^{-1}\left\{f\left(s\right)\right\}+b\cdot L^{-1}\left\{g\left(s\right)\right\}

\begin{equation*}
    =L^{-1}\left\{\frac{20}{103\left(s+3\right)}\right\}-L^{-1}\left\{\frac{7}{54\left(s+4\right)}\right\}+\frac{5203}{5562}L^{-1}\left\{\frac{s+1}{\left(s+1\right)^2+99}\right\}
\end{equation*}
    
\begin{equation*}
    -\frac{1}{1854}L^{-1}\left\{\frac{1}{\left(s+1\right)^2+99}\right\}
\end{equation*}

\begin{equation*}
    =\frac{20}{103}e^{-3t}-\frac{7}{54}e^{-4t}+\frac{5203}{5562}e^{-t}\cos \left(3\sqrt{11}t\right)
\end{equation*}
\begin{equation*}
    -\frac{1}{1854}e^{-t}\frac{1}{3\sqrt{11}}\sin \left(3\sqrt{11}t\right)
\end{equation*}
\end{frame}
\begin{frame}{Solution}
\begin{equation*}
    =\frac{20}{103}e^{-3t}-\frac{7}{54}e^{-4t}+\frac{5203}{5562}e^{-t}\cos \left(3\sqrt{11}t\right)-\frac{1}{5562\sqrt{11}}e^{-t}\sin \left(3\sqrt{11}t\right)
\end{equation*}


\therefore \textbf{Inverse}\:\textbf{Laplace}\:\textbf{Transform}\:\textbf{of} \\
\begin{equation*}
    \frac{(s+5) (s^2+3s+10)}{(s+3)(s+4)(s^2+2s+100)} :
\end{equation*}
\begin{equation*}
    =\frac{20}{103}e^{-3t}-\frac{7}{54}e^{-4t}+\frac{5203}{5562}e^{-t}\cos \left(3\sqrt{11}t\right)-\frac{1}{5562\sqrt{11}}e^{-t}\sin \left(3\sqrt{11}t\right)
\end{equation*}
\end{frame}

\begin{frame}{b. Solution}
\subsection{Solution for b.}
Now let us take function b. \quad i.e.
\begin{equation}
    G(s) = \frac{(s^3+4s^2+2s+6)}{(s+8)(s^2+8s+3)(s^2+5s+7)}
\end{equation}
Inverse Laplace transform of given equation is:
\begin{equation}
    L^{-1}\left\{\frac{s^3+4s^2+2s+6}{\left(s+8\right)\left(s^2+8s+3\right)\left(s^2+5s+7\right)}\right\}
\end{equation}
Now by taking the partial fractions of above eqn, it is
\begin{equation*}
    =L^{-1}\left\{-\frac{266}{93\left(s+8\right)}+\frac{1199s+534}{417\left(s^2+8s+3\right)}+\frac{-65s-1014}{4309\left(s^2+5s+7\right)}\right\}
\end{equation*}
\end{frame}
\begin{frame}{Solution}
\begin{equation*}
    =L^{-1}\left\{-\frac{266}{93\left(s+8\right)}+\frac{1199}{417}\cdot \frac{s+4}{\left(s+4\right)^2-13}-\frac{4262}{417}\cdot \frac{1}{\left(s+4\right)^2-13}
\end{equation*}
\begin{equation*}
    +\frac{-65s-1014}{4309\left(s^2+5s+7\right)}\biggr\}
\end{equation*}
\begin{equation*}
    =L^{-1}\left\{-\frac{266}{93\left(s+8\right)}+\frac{1199}{417}\cdot \frac{s+4}{\left(s+4\right)^2-13}-\frac{4262}{417}\cdot \frac{1}{\left(s+4\right)^2-13}-
\end{equation*} 
\begin{equation*}
     \frac{65}{4309}\cdot \frac{s+\frac{5}{2}}{\left(s+\frac{5}{2}\right)^2+\frac{3}{4}}-\frac{1703}{8618}\cdot \frac{1}{\left(s+\frac{5}{2}\right)^2+\frac{3}{4}}\biggr\}
\end{equation*}
\mathrm{Using\:the\:linearity\:property\:of\:Inverse\:Laplace\:Transform:} \\
\mathrm{For\:functions\:}f\left(s\right),\:g\left(s\right)\mathrm{\:and\:constants\:}a,\:b:\quad L^{-1}\left\{a\cdot f\left(s\right)+b\cdot g\left(s\right)\right\}=a\cdot L^{-1}\left\{f\left(s\right)\right\}+b\cdot L^{-1}\left\{g\left(s\right)\right\}
\end{frame}
\begin{frame}{Solution}
\begin{equation*}
    =-L^{-1}\left\{\frac{266}{93\left(s+8\right)}\right\}+\frac{1199}{417}L^{-1}\left\{\frac{s+4}{\left(s+4\right)^2-13}\right\}
\end{equation*}
\begin{equation*}
    -\frac{4262}{417}L^{-1}\left\{\frac{1}{\left(s+4\right)^2-13}\right\}-\frac{65}{4309}L^{-1}\left\{\frac{s+\frac{5}{2}}{\left(s+\frac{5}{2}\right)^2+\frac{3}{4}}\right\}
\end{equation*}

\begin{equation*}
    -\frac{1703}{8618}L^{-1}\left\{\frac{1}{\left(s+\frac{5}{2}\right)^2+\frac{3}{4}}\right\}
\end{equation*}
\begin{equation*}
    =-\frac{266}{93}e^{-8t}+\frac{1199}{417}e^{-4t}\cosh \left(\sqrt{13}t\right)-\frac{4262}{417}e^{-4t}\frac{1}{\sqrt{13}}\sinh \left(\sqrt{13}t\right)
\end{equation*}
\begin{equation*}
    -\frac{65}{4309}e^{-\frac{5t}{2}}\cos \left(\frac{\sqrt{3}t}{2}\right)-\frac{1703}{8618}\cdot \frac{2}{\sqrt{3}}e^{-\frac{5t}{2}}\sin \left(\frac{\sqrt{3}t}{2}\right)
\end{equation*}
\end{frame}
\begin{frame}{Solution}
\begin{equation*}
    =-\frac{266}{93}e^{-8t}+\frac{1199}{417}e^{-4t}\cosh \left(\sqrt{13}t\right)-\frac{4262}{417\sqrt{13}}e^{-4t}\sinh \left(\sqrt{13}t\right)
\end{equation*}
\begin{equation*}
    -\frac{65}{4309}e^{-\frac{5t}{2}}\cos \left(\frac{\sqrt{3}t}{2}\right)-\frac{1703e^{-\frac{5t}{2}}\sin \left(\frac{\sqrt{3}t}{2}\right)}{4309\sqrt{3}}
\end{equation*}
\therefore \textbf{Inverse}\:\textbf{Laplace}\:\textbf{Transform}\:\textbf{of}: \\
\begin{equation*}
    \frac{(s^3+4s^2+2s+6)}{(s+8)(s^2+8s+3)(s^2+5s+7)} \quad is
\end{equation*}
\begin{equation*}
    =-\frac{266}{93}e^{-8t}+\frac{1199}{417}e^{-4t}\cosh \left(\sqrt{13}t\right)-\frac{4262}{417\sqrt{13}}e^{-4t}\sinh \left(\sqrt{13}t\right)
\end{equation*}
\begin{equation*}
    -\frac{65}{4309}e^{-\frac{5t}{2}}\cos \left(\frac{\sqrt{3}t}{2}\right)-\frac{1703e^{-\frac{5t}{2}}\sin \left(\frac{\sqrt{3}t}{2}\right)}{4309\sqrt{3}}
\end{equation*}
\end{frame}

\section{Code}

%\begin{frame}
%\frametitle{Introduction}
%\framesubtitle{Literature}
%%\begin{figure}[t!]
%%    \centering
%%    \begin{subfigure}[t]{0.4\columnwidth}
%%        \centering
%%        \includegraphics[width=\columnwidth]{point_source}
%%        \caption{Single point source}
%%\label{fig3:subfig1}        
%%    \end{subfigure}%
%%    ~ 
%%    \begin{subfigure}[t]{0.4\columnwidth}
%%        \centering
%%        \includegraphics[width=\columnwidth]{pointNoPowerDist_new}
%%        \caption{SNR profile}
%%\label{fi%%g3:subfig2}
%%    \end{subfigure}
%%  %  \caption{Average SNR for a BPP. $N=16$}
%%    \label{fig3}
%%  \end{figure}
%
%\end{frame}
%  
%
%

\subsection{Python}
\begin{frame}
\frametitle{Python Code for Function a.}
\lstinputlisting[language=python,
firstline=1,lastline=15,
basicstyle=\small]{a.py}

\end{frame}
\begin{frame}[fragile]
\frametitle{Terminal a : }
{\footnotesize


%
\begin{figure}
\centering
\includegraphics[width=1\columnwidth]{./a.png}

\label{fig:plane}
\end{figure}
\end{frame}
\begin{frame}
\frametitle{Python Code for Function b.}
\lstinputlisting[language=python,
firstline=1,lastline=14,
basicstyle=\small]{b.py}
\end{frame}
\begin{frame}
\frametitle{Python Code for Function b.}
\lstinputlisting[language=python,
firstline=16,lastline=25,
basicstyle=\small]{b.py}
\end{frame}
\begin{frame}
\frametitle{Python Code for Function b.}
\lstinputlisting[language=python,
firstline=27,lastline=38,
basicstyle=\small]{b.py}
\end{frame}

\begin{frame}[fragile]
\frametitle{Terminal b : }
{\footnotesize


%
\begin{figure}
\centering
\includegraphics[width=1.02\columnwidth]{./b.png}

\label{fig:plane}
\end{figure}
\end{frame}


\end{frame}
\end{document}
